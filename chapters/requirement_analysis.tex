\chapter{Implementation Plan}
 \section{SOFTWARE REQUIREMENTS}
 \subsection{Python}
Python\cite{python.org} is one of the widely used general-purpose programming language that emphasizes on code readability with its use of significant indenatation. It is an interpreted, high level language that supports both procedural and object oriented paradigm. In 1991 released by Guido van Rossum as Python 0.9.0 which then evolved to Python 2.0 in 2000 and now into Python 3.10.\cite{wikipedia_python}. It is an open source high level interpreted programming language used for Artificial Intelligence, Machine Learning, Deep Learning, Image Processing, Web development, app development e.t.c. \\
\subsection{PostgreSQL}
PostgreSQL \cite{wikipedia_postgresql} is an open-source relational database management system that emphasizes on extensibility and SQL compliance. It supports SQL and is mostly used for app and web development and is frequently used by Python applications for data storage and recovery. It has ACID properties and is evolved from Ingres project. It is considered most advanced and powerful SQL RDBMS that uses multi-version concurrency control(MVCC) which allows several readers to work on the system at once. Due to its capabiltity of handling complex queries it can also be used for data warehousing and data analysis. \\
\subsection{Keras}
asdasdasdadada\\
\subsection{TensorFlow}
asdasdsadadadad
\newpage
\section{FUNCTIONAL REQUIREMENTS}
Functional requirements are the requirements specified by the end user as per the facilities provided by the system. These requirements define the main functionality of the system or one of its subcomponents and describes about the services the software or program should provide to the user. Our system includes the following functional requirements.
\subsection{Number Plate Recognition}
This system tracks number-plate and recognizes vehicle-number form the image. 
\subsection{Number plate specification}
It is system to categorize vehicle depending on number-plate colour.

\subsection{Number plate tracking and recording}
This system record vehicle number along with its passage time and location.

\subsection{Vehicle counting}
This system counts vehicles passing by a road within set time period.\\\\

\section{NON FUNCTIONAL REQUIREMENTS}
Non Functional Requirements are the requirements that specify criteria that can be used to judge the operation of a system rather than specific behaviors of the system. They serve as constraints on design of the system and define system attributes like security, reliability, performance, maintainability, scalability, usability e.t.c.
\subsection{Security}
\lipsum[1]\\
\subsection{Reliability}
\lipsum[1]\\
\subsection{Performance}
\lipsum[1]\\
\subsection{Maintainability}
\lipsum[1]\\
\subsection{Scalability}
\lipsum[1]\\
\subsection{Usability}
\lipsum[1]\\
\subsection{Portability}
\lipsum[1]
\newpage
\section{FEASIBILITY STUDY}
The feasibility of this system can be studied under following points.
\subsection{Economic Feasibility}
\lipsum[1]\\
\subsection{Technical Feasibility}
\lipsum[1]\\
\subsection{Operational Feasibility}
\lipsum[1]
\newpage
