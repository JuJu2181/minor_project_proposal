\chapter{Literature Review}
Automatic Number Plate Recognition system is in state of research and development but in context of Nepal due to lack of proper dataset only small number of researches have been made. In past the systems proposed for ALPR were mostly based on image binarization or gray-scale analysis. However with the rise of DL, Automatic Number Plate Recognition(ANPR) systems have been developed using deep learning techniques like Support Vector Machine(SVM), Convolutional Neural Network(CNN), Deep Neural Network(DNN) e.t.c along with good image processing algorithms for number plate localization and character segmentation. In this project, we planned to use CNN due to its high accuracy for generic object detection and recognition in past.\\\\
The paper "Automatic Nepali Number Plate recognition with Support Vector Machines" by Pant et al. \cite{pant2015automatic} used support vector machines for character recognition and achieved overall accuracy of 75\%. However due to use of incomplete dataset this system may fail to recognize number plates of all the zones. Manish K. Sharma and Bidhan Bhattarai \cite{sharma2017optical} proposed a system that performs optical character recognition for Devnagiri characters in Nepali language by using CNN(Convolutional Neural Network). This system translates images of handwritten or printed texts to machine editable electronic texts and achieved an accuracy of 96\%. This system was not only able to recognize basic characters but also characters obtained from combination of vowels, consonants and special symbols. However this system hasn't been tested for recognizing characters from a number plate and for recognizing different fonts. The paper entitled "LPRNet: License Plate Recognition via Deep Neural Networks" by Sergey Zherdev and Alexey Gruzdev \cite{zherzdev2018lprnet} proposes an end to end method for Automatic License Plate recognition without preliminary character segmentation. They proposed a segmentation free model based on variable length sequence decoding driven by CTC loss also utilizing methods like CNN, RNN, STN. Devising an algorithm that uses a single deep neural network to solve both license plate detection and license plate recognition problems they achieved accuracy of 95\% on Chinese license plates. The paper by Sergio Silva and Claudio Rosito Jung \cite{silva2018license} proposed a complete ALPR system focusing on unconstrained capture scenarios. This system introduces a novel Convolutional Neural Network (CNN) that is capable of detecting and rectifying multiple distorted license plates in a single image which are then fed to OCR for final result and achieved accuracy of 96\% on Taiwanese license plates which is quite fascinating.\\
The paper entitled "Real-Time Bangla License Plate Recognition system using Faster R-CNN and SSD: A Deep Learning Application" \cite{islam2019real} by Tarquil Islam and Risul Islam Rasel uses deep convolutional neural network for license plate detection and characters recognition from a live video stream in real time. This system uses two seperate DCNN models one for detecting the license plate from live video stream and another for recognizing the characters on detected license plate. Trained on the dataset BanglaLekha-Isolated which is similar to Devnagari characters this system achieves 100\% precision on detecting license plate and 91.67\%  precision for detecting characters on the license plate. The paper by Arun Vaishnav and Manju Mandot \cite{vaishnav2018integrated} shows the use of template matching algorithm for recognizing multi language fonts. This sytem can recognize both English as well as Devnagari characters and achieved segmentation accuracy of 97\% and recognition accuracy of 98\%.A recent paper by Pustokhina et al. \cite{pustokhina2020automatic} showed the use of Optimal K-Means with Convolutional Neural Network and achieved a very high accuracy of 98\%. They proposed a novel deep learning based model using Optical K-means(OKM) clustering based segmenation and CNN based recognition called OKM-CNN model. This system operates on 3 stages namely LP localization and detection using Improved Bernsen Algorithm(IBA) and Connected Component Analysis(CCA) models then OKM clustering with Krill Herd(KH) algorithm for image segmentation and finally LP recognition with CNN model. By using 3 different datasets the authors have highly increased the accuracy of the system. They have also evaluated the comparitive study of their approach with other methods and showed effective performance of the OKM-CNN model over other methods.
\\\\
There are numerous methods that can be used for number plate detection and recognition with each method having its own merits and demerits. So in our system we are planning to use a deep learning approach using Convolutional Neural Network for Devnagari character recognition as this appraoch has been promising and has yielded good performance and accuracy in past systems. We have decided to use the Nepali number plate dataset available from github\cite{anpr_dataset_github} and will also try to add our own data later. 